% Introduction and/or Motivation
Analyzing the mechanical structure of different stories, finding the failing reason when an earthquake attacks a building and investigating the impact of different physical attributes of various earthquakes has been a feature of human survival dating back to 230 BC, ancient Chinese recorded the first earthquake in Puzhou. In efforts to build safer building, understanding earthquake simulation data, for examle, the shear stress of each story at different time stamps which an earthquake produce when it is shaking a building, is a long-time pursuing for civil engineers. 

Nevertheless, currently such tasks are not fulfilled perfectly because of the high complexity of the simulation data and the fact that previous work mainly focus on the earthquake itself instead of the simulation data. From visualization perspective, how to visualize such multivariate, multi-scale temporal data remains unclear and potential. In this work, using D3 library~\cite{Bostock:2011:DDD:2068462.2068631},we'd like to design a bunch of visualization views to give us a better understanding of the simulation data and futhermore, compare different earthquake simulation data. 

% Maybe you want to use a list:
The aims of this research are:
\begin{itemize}
  \item design visualization views to help civil enigeering analyze earthquake simulation data
  \item for one earthquake simulation , study the approach to compare impact of different attributes on different stories
  \item compare different earthquake simulations, study the methods for at different levels(i.e attribute level, story level, earthquake level)
\end{itemize}

%The rest of the paper isSection~\ref{sec:vis}