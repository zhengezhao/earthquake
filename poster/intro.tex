% Introduction and/or Motivation
Analyzing the mechanical structure of different building stories, finding the failing reason when an earthquake attacks a building and investigating the impact of different physical attributes produced by various earthquakes has been a feature of human survival dating back to 230 BC, when ancient Chinese recorded the first earthquake in Puzhou. In efforts to build safer building, understanding earthquake simulation data, for example, the shear stress of each story along the entire time stamps which an earthquake generates when it is shaking a building, is a long-time pursuing for civil engineers. 

Nevertheless, currently such tasks can't not be handled and completed perfectly because of the high complexity of the simulation data and the fact that previous work mainly focus on the earthquake itself instead of the simulation data. From visualization perspective, how to visualize such multivariate, multi-scale temporal data remains unclear and potential. In this work, using D3 library~\cite{Bostock:2011:DDD:2068462.2068631},we are trying to design a bunch of visualization views which could help me observe and explore the intricate dataset from different perspectives. Furthermore, we also tried to apply NLP model and kernel methods to these data cubes for comparing them from earthquake level which will be a huge help for the future classification and other machine learning tasks.

Besides, the ais of this research is that we also want to compare these earthquake simulation data from different levels. For example, comparing different attributes, what the relation between Shear and Diaphragm Force (two different forces changed during earthquake ) when the earthquake happens. And also what's the behaviour of shear stress among different stories.

% Maybe you want to use a list:

%The rest of the paper isSection~\ref{sec:vis}