\section{Related Work}
\label{sec:related}
The problem of building shaking during an earthquake has been studied for years, in order to investigate the performance of constructions under a shake table, in 2007, the engineers build a seven-story building for testing and  re-creation of a seismic response of a seven-story building based on recorded data of a full scale shake table test.  ~\cite{Chourasia:2007:DRS:1247238.1247243} Other than this, as far as we know, most of the work focus on recreate earthquake based data collected and visualizing large-scale influence of a real earthquake. Fewer work has been done to study relationship between different attrbute when the earthquake is shaking the building.

In the meanwhile,exploration of multivariate data sets is an integral part of scientific visualization as in most real world phenomena, there exist multiple factors associated with the complex interactions of different variables. To gain an in-depth understanding of a scientific process, the relationship among the variables needs to be thoroughly investigated. Biswas et al.~\cite{Biswas:2013:AIFEMDS:1077-2626} propose a framework to classify isocontours of variables based on the relationship between them and users can explore the multivariate data sets using their interface. Comparing time series data is another greate challenge, researchers apply different kernel methods to study the correlation of different time series and Vert et al. propose a alignment kernel for time series. ~\cite{DBLP:journals/corr/abs-cs-0610033} 


We heavily use the well-known d3 library~\cite{Bostock:2011:DDD:2068462.2068631} to create our tool.
