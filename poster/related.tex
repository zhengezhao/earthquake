\section{Related Work}
\label{sec:related}
The problem of how a building shaking during an earthquake has been studied for years.
Howerve, as far as we know, most of the work focus on recreate earthquake based data collected and visualizing large-scale influence of a real earthquake.In order to investigate the performance of building on a shake table, in 2007, the engineers built a seven-story building for testing and  re-creation of a seismic response of it  based on recorded data of a full scale shake table test.  ~\cite{Chourasia:2007:DRS:1247238.1247243} Fewer work has been done to study relationship between different attributes when the earthquake is shaking the building.

In the meanwhile,exploration of multivariate data sets is an integral part of scientific visualization. As in most real world phenomena, there exist multiple factors associated with the complex interactions of different variables. To gain an in-depth understanding of a scientific process, the relationship among the variables needs to be thoroughly investigated. Biswas et al.~\cite{Biswas:2013:AIFEMDS:1077-2626} propose a framework to classify isocontours of variables based on the relationship between them and users can explore the multivariate data sets using their interface. Comparing time series data is another greate challenge. Kernel methods, such as Support Vector Machines, Gaussian Process have been proved to be extremely success. Vert et al. propose an alignment kernel for time series which is widely used when comparing time series. ~\cite{DBLP:journals/corr/abs-cs-0610033} 

